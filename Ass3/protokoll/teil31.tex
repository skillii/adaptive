\chapter{Analytical Problem 3.1: Predictive Encoding of an AR Process}

\paragraph{a)}

Die Autokorrelationsfaktoren ergeben sich wie folgt:

Dabei verschwindet $E\{w[n]u[n-k]\}$ f"ur $k>0$, da das weisse Rauschen unkorreliert mit $u[n-k]$ f"ur $k>0$ ist!

\begin{align}
r_{uu}[k] &= E\{u[n]u[n-k]\}\\
&= E\{(w[n]+u[n-1]-1/8u[n-2])((w[n-k]+u[n-1-k]-1/8u[n-2-k])\} \\
&= E\{\underbrace{w[n]u[n-k]}_0 + \underbrace{u[n-1]u[n-k]}_{r_{uu}[k-1]} - 1/8\underbrace{u[n-2]u[n-k]}_{r_{uu}[k-2]}
\end{align}

\begin{align}
r_{uu}[0] &= E{u^2[n]} = \sigma_u^2 = 1 \\
r_{uu}[1] &= r_{uu}[0] - 1/8\underbrace{r_{uu}[1-2]}_{=r_{uu}[1]} = 8/9 \\
r_{uu}[2] &=  r_{uu}[1] - 1/8r_{uu}[0] = 55/72
\end{align}

\begin{align}
 r_{uu} &= E\{(w[n]+u[n-1]-1/8u[n-2])^2\} \\
 &= E\{\underbrace{w^2[n]}_{\sigma_w^2}+\underbrace{w[n]u[n-1]}_{0}-1/8\underbrace{w[n]u[n-2]}_{0} \\
 &+\underbrace{w[n]u[n-1]}_{0} + \underbrace{u^2[n-1]}_{\sigma_u^2} -1/8\underbrace{u[n-1]u[n-2]}_{r_{uu}[1]} \\
 &-1/8\underbrace{w[n]u[n-2]}_{0} - 1/8\underbrace{u[n-1]u[n-2]}_{r_{uu}[1]} + 1/64\underbrace{u^2[n-2]}_{\sigma_u^2} \} \\
r_{uu}[0] &= \sigma_u^2 = \sigma_w^2 + \sigma_u^2 - 2/8r_{uu}[1] + 1/64\sigma_u^2 \\
 \Rightarrow \sigma_r &= 2/9 - 1/64 = 119/576
\end{align}

Der Noisegain $G_G$ berechnet sich wie folgt:

\begin{equation}
 G_G = \frac{\sigma_u^2}{\sigma_w^2} = 576/119
\end{equation}

\paragraph{b)}

Die Jule Walker Gleichung ist wie folgt definiert:

\begin{equation}
 \underline{c}_{MSE} = R_{uu}^{-1} \cdot \begin{bmatrix} r_{uu}[1] \\ r_{uu}[2] \\ ... \\ r_{uu}[N]\end{bmatrix}
\end{equation}

f"ur N=1 ergibt sich somit f"ur $c_{MSE}$

\begin{equation}
 c_{MSE} = c_0 = \frac{1}{r_{uu}[0]} \cdot r_{uu}[1] = 8/9
\end{equation}


Um nun die Varianz des Fehlers $e[n]$ zu berechnen  wird wieder einfach in $E\{e^2[n]\}$ eingesetzt:

\begin{align}
 e[n] &= u[n] - u[n-1] \cdot 8/9 \\
 &= w[n] + 1/9 u[n-1] - 1/8 u[n-2]
\end{align}

\begin{align}
 E\{e^2[n]\} &= E\{\underbrace{w^2[n]}_{\sigma_w^2} + 1/8 \underbrace{u[n-2]w[n]}_{0} - 1/9\underbrace{w[n]u[n-1]}_{0} + 1/81\underbrace{u^2[n-1]}_{\sigma_u^2} \\
 &- 1/72\underbrace{u[n-1]u[n-2]}_{r_{uu}[1]} - \underbrace{w[n]1/8u[n-2]}_{0} - 1/72\underbrace{u[n-1]u[n-2]}_{r_{uu}[1]}  \\
&+ 1/64\underbrace{u^2[n-2]}_{\sigma_u^2}\} \\
&= \sigma_w^2 + 1/81\sigma_u^2 - 1/36r_{uu}[1] + 1/64\sigma_u^2 \\
&= 17/81
\end{align}

Der Prediction Gain $G_C$ berechnet sich wie folgt:

\begin{equation}
 G_C = \frac{\sigma_u^2}{\sigma_e^2} = 81/17
\end{equation}


Der verwendete lineare Predictor ist nicht optimal, da das Signal $u[n]$ von einem AR prozess
stammt der aus Weissem Rauschen mit einem Filter 1. Ordnung erzeugt wurde. Somit reicht ein
linearer Predictor 1. Ordnung nicht aus. 
Am Signal $e[n]$ erkennt man auch, dass das Signal noch kein weisses rauschen ist.


\paragraph{c)}

F"ur N=1 wurde die gleiche Vorhergehensweise verwendet:

\begin{equation}
 \underline{c}_{MSE} = R_{uu}^{-1} \cdot \begin{bmatrix} r_{uu}[1] \\ r_{uu}[2] \\ ... \\ r_{uu}[N]\end{bmatrix}
 = \begin{bmatrix} 1 & 8/9 \\ 8/9 & 1\end{bmatrix}^-1 \cdot \begin{bmatrix}8/9 \\55/72\end{bmatrix}
 = \begin{bmatrix}1 \\ -1/8\end{bmatrix}
\end{equation}

\begin{align}
 e[n] = u[n] - 1\cdot u[n-1] + 1/8 \cdot u[n-2] = w[n]
\end{align}

\begin{equation}
 G_C = \frac{\sigma_u^2}{\sigma_e^2} = \frac{1}{\sigma_w^2} = \frac{576}{119}
\end{equation}


Der verwendete Predictor ist nun optimal, da er aus dem AR prozess wieder das weisse rauschen gewinnt.
Eine bessere Vorhersage ist nicht mehr m"oglich, da weisses Rauschen ja nicht vorherhersehbar ist.

\paragraph{d)}

N=1: der Prediction Error filter besitzt folgende Differenzengleichung:
\begin{equation}
 e[n] = u[n] - 8/9 u[n-1]
\end{equation}

Somit ergibt sich f"ur die "Ubertragungsfunktion:

\begin{equation}
 H_e(z) = 1-8/9z^{-1}
\end{equation}

F"ur $S(z)$ wird einfach das Inverse von $H_e(z)$ ermittelt:

\begin{equation}
 S(z) = \frac{1}{1-8/9z^{-1}}
\end{equation}

Um den Noisegain zu ermitteln wird das Quadrat der Norm der Impulsantwort des Filters $S(z)$ ermittelt:

\begin{equation}
 G_s = ||g_s[n]||^2 = \sum^{\inf}_{i=0} (8/9)^{2n} = \frac{1}{1-64/81} = \frac{81}{17}
\end{equation}


Da es sich bei dem Input des Filters S(z) um kein weisses Rauschen handelt, ist keine Berechnung
von $\sigma_{\hat{u}}^2$ m"oglich. ($e[n]$ ist kein weisses Rauschen!)


N=2: der Prediction Error filter besitzt folgende Differenzengleichung:
\begin{equation}
 e[n] = u[n] - u[n-1] - u[n-1] + 1/8 \cdot u[n-2]
\end{equation}
Somit ergibt sich f"ur die "Ubertragungsfunktion:
\begin{equation}
 H_e(z) = 1-z^{-1} + 1/8z^{-2}
\end{equation}

F"ur $S(z)$ wird einfach das Inverse von $H_e(z)$ ermittelt:

\begin{equation}
 S(z) = \frac{1}{1-z^{-1} + 1/8z^{-2}}
\end{equation}


F"ur die Ermittelung des Noisegains wurde die Tatsache ausgenutzt, dass der Filter $S(z)$ genau
das Inverse von $H_e(z)$ darstellt. Der Noisegain von $H_e(z)$ wurde bereits in Aufgabe 3.1.a) ermittelt.

Dadurch ergibt sich nun folgender Noisegain:

\begin{equation}
 G_s = \frac{576}{119}
\end{equation}

Da bei N=1 das Eingangssignal des Filters $S(z)$ weisses Rauschen ist, ist eine Berechnung von 
$\sigma_{\hat{u}}^2$ mittels $G_s$ m"oglich.



\paragraph{e)}

Um das Signal $\hat{u}[n]$ verwenden wir zun"achst eine Darstellung im Z-Bereich. Weiters 
verwenden wir auch die Tatsache, dass f"ur die Ermittelung von $S(z)$ das Inverse des Prediction-error-filters
verwendet wurde. Somit ist:
\begin{equation}
S(z) = \frac{U(z)}{E(z)} 
\end{equation}


\begin{align}
\hat{U}[n] &= \hat{E}(z) \cdot S(z)\\
 &= (E(z) + Q(z)) \cdot g \cdot S(z) \\
 &= (E(z) + Q(z)) \cdot g \cdot \frac{U(z)}{E(z)}  \\
 &= (U(z) + Q(z) \cdot S(z)) \cdot g
\end{align}

Wieder in den Zeitbereich zur"ucktransformiert ergibt das:

\begin{equation}
 \hat{u}[n] = g \cdot u[n] + g \cdot q[n] * s[n]
\end{equation}

\begin{equation}
 r[n] = \hat{u}[n] - u[n] = (1-g)u[n] + g \cdot q[n] * s[n]
\end{equation}

\begin{align}
 E\{r^2[n]\} &= E\{(1-g)^2 u^2[n] + \underbrace{2(1-g)u[n]g\cdot q[n] * s[n]}_{=0, da unkorreliert} + \underbrace{(g \cdot q[n]*s[n])^2}_{g^2v\sigma_q^2v||s[n]||^2}\} \\
 &= (1-g)^2 \cdot \sigma_u^2 + g^2\sigma_q^2 ||s[n]||^2 \\
 &= (1-g)^2 \cdot \sigma_u^2 + g^2\sigma_q^2 G_c
\end{align}


\begin{equation}
 \sigma_q^2 = \frac{\sigma_e^2}{2^{2R}-1} = \sigma_u^2/G_c \cdot \frac{1}{2^{2R}-1}
\end{equation}


\begin{equation}
 g = \frac{\sigma_e^2}{\sigma_e^2 + \sigma_q^2} = \frac{\sigma_u^2/G_C}{\sigma_u^2/G_C + \sigma_u^2/G_c\frac{1}{2^{2R}-1}}
 = \frac{1}{1+\frac{1}{2^{2R}-1}} = \frac{2^{2R} - 1}{2^{2R}}
\end{equation}

\begin{align}
 \sigma_r^2 &= [(\frac{2^{2R} - 1}{2^{2R}} - 1)^2 + \frac{(2^{2R} - 1)^2}{2^{4R}} \frac{1}{G_c} \frac{G_c}{2^{2R} - 1}]\sigma_u^2 \\
 &= [(\frac{-1}{2^{2R}})^2 +  \frac{2^{2R} - 1}{2^{4R}}]\sigma_u^2 \\
 &= \frac{\sigma_u^2}{2^{2R}}
\end{align}


\paragraph{f)}
Bei einer Bitrate von 1 Bit / Sample ergibt sich nun folgende Varianz von $r[n]$ bei N=1 und N=2:

\begin{equation}
 \sigma_r^2 = \frac{\sigma_u^2}{4}
\end{equation}

Nun wird Die Varianz von $r[n]$ bei direkter Codierung berechnet.
Um die Varianz von $r[n]$ zu berechnen wurde ein formel aus der Signalverarbeitungs-VO f"ur den SNR
bei einer quantisierung verwendet:

\begin{equation}
 SNR = \frac{\sigma_u^2}{\sigma_q^2} = 2^2 = 4
\end{equation}
Somit ergit sich f"ur $\sigma_q^2$:
\begin{equation}
 \sigma_q^2 = \frac{\sigma_u^2}{4}
\end{equation}

\begin{equation}
 r[n] = \hat{u}[n] - u[n] = q[n]
\end{equation}

\begin{equation}
 \sigma_r^2 = \sigma_q^2 = \frac{\sigma_u^2}{4}
\end{equation}

Somit ergibt sich f"ur direkte Codierung und linear Predictive Coding der Die gleiche varianz
des Rekonstruktionsfehlers. => kein Gain bei der Verwendung von Prediction!










