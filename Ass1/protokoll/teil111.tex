\chapter{Analytic Problem 1.1}

\subsection{(a)}
Zuerst stellen wir eine Kostenfunktion $J(c)$, wie in den Problem-classes auf.

\begin{equation}
 J(c) = E\{e[n] \cdot e[n]\}
\end{equation}

Um das Minimum der Kostenfunktion zu ermitteln, bilden wir den Gradienten mit den Ableitungen nach den
einzelnen Filterkoeffizienten $\textbf{c}$.

\begin{equation}
 \bigtriangledown J(c) = E\{\bigtriangledown e[n] \cdot e[n]\}
 \label{eq:kosten1}
\end{equation}

Setzt man die Beziehung $e[n] = d[n] - \textbf{c}^T \textbf{x}[n]$ in die Gleichung \ref{eq:kosten1} ein,
so erh�lt man:

\begin{equation}
 \bigtriangledown J(c) = E\{(\bigtriangledown d[n] - \bigtriangledown \textbf{c}^T \textbf{x}[n]) \cdot e[n]\}
\end{equation}

\begin{equation}
 \bigtriangledown J(c) = E\{-\textbf{x}[n] \cdot e[n]\}
\end{equation}

Um nun das Minimum der Kostenfunktion zu erhalten m�ssen wir diesen Term 0 setzen:

\begin{equation}
 \bigtriangledown J(c) = E\{-\textbf{x}[n] \cdot e[n]\} \stackrel{!}{=} 0
 \label{eq:aresult}
\end{equation}

Hiermit ist ersichtlich, dass jedes $x[n-k]$ orthogonal zu $e[n]$ ist.


\subsection{(b)}

Um diese Beziehung herzuleiten verwenden wir die Gleichung \ref{eq:aresult} und multiplizieren
dise von links mit $\textbf{c}^T$.
Somit erhalten wir:

\begin{equation}
 E\{-\textbf{c}^T \textbf{x}[n] \cdot e[n]\} \stackrel{!}{=} 0
\end{equation}

\begin{equation}
 E\{-y[n] \cdot e[n]\} \stackrel{!}{=} 0
\end{equation}

Somit ist auch $y[n]$ orthogonal zu $e[n]$


\subsection{(c)}

Um den Wert f�r die Kostenfunktion bei optimalen Filterkoeffizienten zu berechnen, setzen wir einfach
die Wiener Hopf Solution in die Gleichung f�r die Kostenfunktion (Gleichung \ref{eq:kosten2}).

\begin{equation}
 J(\textbf{c}) = E\{d^2[n] - 2d[n]\textbf{c}^T\textbf{x}[n] + \textbf{c}^T\textbf{x}[n]\textbf{x}^T[n]\}
\end{equation}

\begin{equation}
 J(\textbf{c}) = E\{d^2[n]} - 2\textbf{c}^T\textbf{p} + \textbf{c}^T\textbf{R}_{xx}\}
 \label{eq:kosten2}
\end{equation}

Somit erhalten wir:

\begin{equation}
 J(\textbf{c}) = \sigma_d - 2\textbf{p}^T\textbf{R}_{xx}\textbf{p} + \textbf{p}^T\textbf{R}_{xx}\textbf{p}
\end{equation}

\begin{equation}
 J(\textbf{c}) = \sigma_d - \textbf{p}^T\textbf{R}_{xx}\textbf{p}
\end{equation}