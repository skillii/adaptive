\clearpage
\chapter{Analytic Problem 1.2}

\paragraph{a)}
Wie in der "Ubung am 25.10.2011 hergeleitet kann die Kostenfunktion wie folgt angeschrieben werden:

\begin{equation}
 J(\vm{c}) = \sigma_v^2 - 2\vm{c}^T \vm{p} + \vm{c}^T\vm{R}_{xx}\vm{c}
\label{eq:cost1}
\end{equation}

Die Gleichung \ref{eq:cost1} kann wie folgt umformuliert werden:

\begin{equation}
 J(\vm{c}) = \sigma_v^2 - \vm{p}^T\vm{R}_{xx}^{-1}\vm{p} + (\vm{c} - \vm{R}_{xx}^{-1}\vm{p})^T\vm{R}_{xx}(\vm{c} - \vm{R}_{xx}^{-1}\vm{p})
\label{eq:cost2}
\end{equation}


Beweis: Durch Ausmultiplizieren der Klammern gelangt man wieder auf die Gleichung \ref{eq:cost1}:

\begin{align}
 J(\vm{c}) &= \sigma_v^2 - \vm{p}^T\vm{R}_{xx}^{-1}\vm{p} + (\vm{c}^T - \vm{p}\vm{R}_{xx}^{-1})(\vm{R}_{xx}\vm{c} - \vm{R}_{xx}\vm{R}_{xx}^{-1}\vm{p}) \\
  &=  \sigma_v^2 - \vm{p}^T\vm{R}_{xx}^{-1}\vm{p} + \vm{c}^T\vm{R}_{xx}\vm{c} - \vm{p}^T\vm{R}_{xx}^{-1}\vm{R}_{xx}\vm{c} - \vm{c}^T\vm{p} + \vm{p}^T\vm{R}_{xx}^{-1}\vm{p} \\
  &=  \sigma_v^2 + \vm{c}^T\vm{R}_{xx}\vm{c} - \vm{p}^T\vm{c} - \vm{c}^T\vm{p} \\
  &=  \sigma_v^2 + \vm{c}^T\vm{R}_{xx}\vm{c} - 2 \vm{c}^T\vm{p}
\label{eq:cost_beweis}
\end{align}

In der Gleichung \ref{eq:cost2} kommt der Ausdruck $\vm{c} - \vm{R}_{xx}^{-1}\vm{p}$ vor. Dieser Ausdruck entspricht
genau dem Misalignment-Vector, da $\vm{R}_{xx}^{-1}\vm{p}$ der Wiener Hopf-Solution entspricht und somit die optimale
L"osung darstellt.

Somit kann die Kostenfunktion in Abh"angigkeit von $\vm{v}$ (Misalignment-Vektor) ausgedr"uckt werden:
\begin{equation}
 J(\vm{c}) = \sigma_v^2 - \vm{p}^T\vm{R}_{xx}^{-1}\vm{p} + \vm{v}^T\vm{R}_{xx}\vm{v}
\label{eq:cost3}
\end{equation}
Anhand dieser Gleichung erkennt man, dass der vordere Teil ($\sigma_v^2 - \vm{p}^T\vm{R}_{xx}^{-1}\vm{p}$)
unabh"angig vom Misalignment-Vektor ist und somit das Minimum der Kostenfunktion($J_{min}$) darstellt.

\paragraph{b)}

Die Autokorrelationsmatrix $\vm{R}_{xx}$ kann mittels der Eigenwerte/Eigenvektoren wie folgt zerlegt werden:

\begin{equation}
\vm{R}_{xx} = \vm{Q}\vm{\Delta}\vm{Q}^T
\end{equation}

Wobei die Matrix $\vm{\Delta}$ eine Diagonalmatrix ist, welche die Eigenwerte von $\vm{R}_{xx}$ enth"alt.
Die Matrix $\vm{Q}$ enth"alt alle Eigenvektoren.

Diese Beziehung kann in die Gleichung \ref{eq:cost3} eingesetzt werden und man erh"alt:

\begin{equation}
 J(\vm{c}) = J_{min} + \vm{v}^T\vm{Q}\vm{\Delta}\vm{Q}^T\vm{v}
\label{eq:costasfd}
\end{equation}

F"ugt man nun noch folgende Substituion ein: $\vm{\tilde{v}} = \vm{Q}^T\vm{v}$, so erh"alt man
eine Gleichung f"ur die Kostenfunktion bei der die einzelnen Komponenten von $\vm{\tilde{v}}$ entkoppelt sind:

\begin{equation}
 J(\vm{c}) = J_{min} + \vm{\tilde{v}}^T\vm{\Delta}\vm{\tilde{v}}
\label{eq:cost4}
\end{equation}

Die Gleichung \ref{eq:cost4} in Matrixschreibweise kann nun wie folgt in eine Summe umgeschrieben werden:

\begin{equation}
 J(\vm{c}) = J_{min} + \sum_{k=1}^N \tilde{v}_k^2 \lambda_k
\label{eq:cost5}
\end{equation}

\paragraph{c)}

Wie in der "Ubung vom 22.11.2011 gezeigt wurde, verhalten sich die einzelnen Komponenten von $\vm{\tilde{v}}$
wie folgt:

\begin{equation}
 \tilde{v}_k[n] = (1-\mu \lambda_k)^n \tilde{v}_k[0] = \tilde{v}_k[0]e^{-n/\tau_k}
\label{eq:misalignment_evolution}
\end{equation}

Diese Gleichung in Gleichung \ref{eq:cost5} eingesetzt ergibt:
\begin{equation}
 J(\vm{c}) = J_{min} + \sum_{k=1}^N \tilde{v}_k^2[0] \lambda_k e^{-2n/\tau_k}
\label{eq:cost6}
\end{equation}

\paragraph{d)}

Die Zeitkonstanten k"onnen aus der Gleichung \ref{eq:misalignment_evolution} ermittelt werden:
Daraus ergibt sich f"ur $\tau_k$ (wie auch in der "Ubung bereits hergeleitet) folgendes:

\begin{equation}
 \tau_k = \frac{-1}{log|1-\mu \lambda_k|}  \approx \frac{1}{\mu \lambda_k}
\end{equation}

\begin{equation}
 J(\vm{c}) = J_{min} + \sum_{k=1}^N \tilde{v}_k^2[0] \lambda_k e^{-2n \mu \lambda_k}
\label{eq:cost7}
\end{equation}

\paragraph{e)}

White noise with unit variance => $R_{xx} = \vm{I}$. Die Eigenwerte einer Diagonalmatrix entsprechen genau
den Werten in der Diagonale. Somit sind beide Eigenwerte = 1.
Die Eigenvektoren sind $[1 0]^T$ und $[0 1]^T$. Die Eigenvektoren in die Matrix $\vm{Q}$ eingetragen ergibt:

\begin{equation}
 \vm{Q}^T = \vm{Q} = \begin{pmatrix} 1 & 0 \\ 0 & 1 \end{pmatrix}
\end{equation}

\begin{equation}
 \vm{\tilde{v}}[0] = \vm{Q}^T \vm{v}[0] = \begin{pmatrix} -2 \\ -1 \end{pmatrix}
\end{equation}

Diese Werte in die Gleichung f"ur die Kostenfunktion eingesetzt ergibt:

\begin{equation}
 J(\vm{c}) = J_{min} + 4e^{-2n\mu} + e^{-2n\mu}
\label{eq:cost_example}
\end{equation}